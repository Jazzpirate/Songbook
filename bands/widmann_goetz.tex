% Götz Widmann

\songtodo{Die Zwei Trauben}
\beginsong{Die Zwei Trauben}
\beginverse
Es hingen einst zwei Trauben, wie die Turteltauben
ungestört an einem Strang, ihr ganzes Sommerleben lang
abends, wenn die Grillen sangen, packte die beiden ein Verlangen,
das es bei Trauben sonst nicht gibt. Sie waren unsterblich verliebt.
Er war ein dunkler warmer Typ, das hatte sie an ihm so lieb.
Zur Nachtigallensymphonie sang er ganz leis allein für sie
0h du meine schöne Traube, glaub mir, dass ich an dich nur glaube.
Du meine wunderschöne Braut mit deiner zarten glatten Haut.
\endverse
\beginchorus
Mein Leben tauscht ich dafür ein, nur einmal eins mit dir zu sein.
Mein Leben tauscht ich dafür ein, nur einmal eins mit dir zu sein.
\endchorus
\beginverse
Sie sprach, ach ja, das wär so schön, nur wird es leider nie geschehn,
anders als Menschen oder Affen sind wir nicht dafür geschaffen.
Du bist nirgends lang und spitz und ich hab nirgendwo nen Schlitz.
Gott wollte uns das nicht erlauben darum schuf ' er uns als Trauben.
Vielleicht ham wir in nem andern Leben uns uns in Sünde hingegeben
ohne sein heilges Wort zu achten und müssen hier jetzt dafür schmachten.
Wir wollten einst nicht an ihn glauben, darum sind wir jetzt nur Trauben.
Wenigstens darf ich dich berührn und dich ganz nah bei mir spürn.
\endverse\textnote{Chorus}
\beginverse
Da kam ein junger Bauersmann, hat lustlos seinen Job getan
alle Trauben abgerissen und in seinen Korb geschmissen.
Unsre beiden zarten Süßen zerplatzten unter seinen Füßen
um doch dann ganz kurz zu genießen sich ineinander zu ergießen.
Das Glück, in dem sie sich verloren ist dann in einem Fass vergoren.
lmmerhin, die beiden starben. indem sie sich einander gaben.

Und ich sitz hier heut nacht allein und trinke meinen roten Wein
denk an die Trauben, an uns zwei und sing ganz leis mein Lied dabei.
\endverse
\textnote{Chorus}
\endsong
